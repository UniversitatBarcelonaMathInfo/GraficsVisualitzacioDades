\subsection{Nomenclatura}
\begin{itemize}
    \item[Gràfics en Computador:]
        Disciplina de la informàtica que estudia \textbf{models} i els \textbf{algorismes}
        implicats en la construcció d'imatges 2D a partir d'un món virtual.

    \item[Visualització 3D interactiva:]
        Descriu qualsevol ús dels computadors per \textbf{crear} o \textbf{interaccionar}
        amb imatges a partir d'un món virtual i des d'un determinat punt de vista.

    \item[FPS:] Frames per segon. Imatges / segons.
        \begin{itemize}
            \item Interacció amb FPS mínima.
                \subitem General: 6 FPS
                \subitem Videojocs: 30 o 60 FPS.
                \subitem Pe\lgem{}ícules: 24 FPS.
        \end{itemize}

    \item[Imatge:] Resultat visualitzada per pantalla.
    \item[Modelatge:] Objectes, material i textures.
        \begin{itemize}
            \item[Objectes:] Representació 3D ``volum''.
            \item[Textures:] Modelen realisme addicional, per detallar característiques.
            \item[Material:] Comportament dels objectes amb la llum.
                \subitem Llum: emissió, absorció i scattering.
            \item[Tècniques:] Visualització i i\lgem{}uminació
        \end{itemize}
    \item[Punt de vista:] Des d'on mires i cap a on.
        \subitem[Càmera:] Punt de vista i expansió de la visió.
    \item[Models 3D:] Objectes que tenen un cert volum.
    \item[Dispositiu gràfic:] !!!!!!!!!!!!!!!
    \item[frame buffer:] on guardem el frame per mostrar seguidament.
    \item[Elements gràfics:] .
        \begin{itemize}
            \item[Objectes:] Volum, materials i textures.
                \item Llum: Objecte que emet llum.
                \begin{itemize}
                    \item[Superficials:] Només coneixem la frontera.
                    \item[Volumètrics:] 
                \end{itemize}
            \item[Càmera:] Posició del observador i frustum.
                \subitem Frustum: Part visible de l'escena.
            \item[Viewport:] Conjunt de píxels on es formarà la imatge final. !!!!!!!!!!!!!!!!!!!!!!!
        \end{itemize}
        
    % Interessant a posar al pdf.
    \item[Pipeline de visualització:] Conjunt d'etapes que segueix en el procés de visualització.
        (i\lgem{uminació}, projecció, \dots)
    \item[Rendering:] Visualització, procés de càlcul del color al píxel final.
        Normalment té una relació amb la i\lgem{uminació} final.
        
    \item[Shaders] GPU. Fils d'execució de la GPU.
\end{itemize}

\subsection{Visualització -- No section...}
\subsubsection{Mètode Projectius}
Zbuffer
\begin{itemize}
    \item Més ràpid. (basat en hardware)
    \item Menys realista.
    \item Fàcil d'implementar amb \textbf{shaders}.
\end{itemize}

\subsubsection{Mètode Ray*}
RayCasting / Raytracing
\begin{itemize}
    \item I\lgem{uminació} global.
    \item Tradicionalment és lent.
    \item Recentment és pot implementar en GPU.
    \item Fàcils d'adaptar per models volumètrics.
\end{itemize}


\subsection{Components software}
Gràfic: teclat -> CPU -> GPU -> Frame buffer -> pantalla
                <-> CPU Memori i el mateix epr al GPU.

tbl
Aplication program  API             Hardware{2}
High                Intermediate    Low     GPU-based
Paraview            Qt              OpenGL  GLSL
Unity               WebGL