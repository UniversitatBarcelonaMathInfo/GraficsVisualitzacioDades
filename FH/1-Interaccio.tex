El disseny de la interacció ha de tenir en compte a l'usuari:
\begin{itemize} % La idea.
    \item Facilita el treball.
    \item Eficient.
    \item Aprés fàcilment.
    \item Errors.
    \item Experiència agradable.
\end{itemize}
% Sinó, tabula el text i no és desitjat aquest resultat
Tenim ajudes:
\begin{itemize}
    \item Guies d'usabilitat. % És correcta aquesta apostrofació.
    \item Mètodes i tècniques de disseny de la interacció.
\end{itemize}

\subsection{\textbf{IU}, La interacció amb l'usuari}
Té un paper fonamental en la interacció humà-ordinador.\\
\textbf{HCI}, Human-computer interaction:
Disciplina interessada en el disseny,
avaluació i implementació de sistemes computacionals interactius per ús humà i en l'estudi dels fenòmens que els rodegen.

\subsection{\textbf{UCD}, Disseny centrat en l'usuari}
És un procés iteratiu: Disseny, Prototipat, Avaluació. % Prototipat, paraula correcta.
\begin{itemize}
    \item Effective `Efectiu'
    \item Efficient `Eficient'
    \item Easy to learn `Fàcil d'aprendre'
    \item Error tolerant `Tolerant als errors'
    \item Engaging `Atractiu'
\end{itemize}
%
Principis generals \textbf{UCD}
\begin{itemize}
    \item Enfocament als usuaris i les seves tasques.
    \item Disseny iteratiu amb prototips.
    \item Abaluació dels prototips.
\end{itemize}
%
Perquè és adequat \textbf{UCD}?
\begin{itemize}
    \item Es té en compte a l'usuari.
    \item Gestiona millor el risc.
    \item L'usuari avalua cada iteració.
    \item Menys codi, només les \textbf{IU} madures.
\end{itemize}

\subsection{Design thinking}
Entendre millor i solucionar les necessitats dels usuaris.
\begin{enumerate}
    \item Empathize: Entendre i conèixer l'usuari. `Observar, escoltar, preguntar'.
    \item (re)Define: Dissenyar.
    \item Ideate: Trobar solucions als problemes.
    \item Prototype: Tenir un prototip amb les noves solucions.
    \item Test: Obtenir un feedback dels usuaris.
\end{enumerate}